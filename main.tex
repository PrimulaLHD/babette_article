\documentclass{bioinfo}
\copyrightyear{2017} \pubyear{2017}

\access{Advance Access Publication Date: Day Month Year}
\appnotes{Manuscript Category}

\begin{document}
\firstpage{1}

\subtitle{APPLICATIONS NOTES}

\title[beastscriptr]{beastscriptr: BEAUti for R}
\author[Sample \textit{et~al}.]{Rich\`el J.C. Bilderbeek\,$^{\text{\sfb 1,}*}$, Co-Author\,$^{\text{\sfb 2}}$ and Co-Author\,$^{\text{\sfb 2,}*}$}
\address{$^{\text{\sf 1}}$TR\^ES, Groningen Institute for Evolutionary Life Sciences, Groningen, 9747 AG, The Netherlands and \\
$^{\text{\sf 2}}$Department, Institution, City, Post Code,
Country.}

\corresp{$^\ast$To whom correspondence should be addressed.}

\history{Received on XXXXX; revised on XXXXX; accepted on XXXXX}

\editor{Associate Editor: XXXXXXX}

\abstract{\textbf{Motivation:} BEAST2 is a popular Bayesian phylogenetics tool.
The program BEAUti allows for a user-friendly way to create the
input files that BEAST2 needs. This is convenient to get started, yet
does not facilitate the scripted use of BEAST2. 
beastscriptr is an R package that can create BEAST2 input files from an R
function call.\\
\textbf{Results:} beastscriptr has only parts of the functionality of BEAUti, yet
can be extended easily. beastcriptr, on the other hand, does allow for specifying
to use phylogenies of fixed crown age.\\
\textbf{Availability:} The package is free and is available from the official R package archive at 
http://cran.r-project.org/src/contrib/PACKAGES.html\#beastscriptr. 
beastscriptr is licensed under the GNU General Public License.
\textbf{Contact:} \href{name@bio.com}{name@bio.com}\\
\textbf{Supplementary information:} Supplementary data are available at \textit{Bioinformatics}
online.}

\maketitle

\section{Introduction}

BEAST2 \cite{bouckaert2014beast} is a popular Bayesian phylogenetics tool.
The program BEAUti \cite{drummond2012bayesian} allows for a user-friendly 
way to create the
input files that BEAST2 needs. This is convenient to get started, yet
does not facilitate the scripted use of BEAST2. 
beastscriptr is an R package that can create BEAST2 input files from an R
function call.

\section{Approach}

The goal of the R package beastscriptr was to mimic the functionality of BEAUti.
Where BEAUti creates a BEAST2 XML input file by a graphical user interface,
beastscriptr would do the same from one R function call.

\begin{methods}
\section{Methods}

By creating BEAST2 input files in BEAUti, the files beastscriptr should
create were determined. BEAUti already has many options, already
before taking all possible plug-ins into account. beastscriptr 
gradually grew in mimicking the basic functionality of BEAUti.

beastscriptr has a novel functionality not built into BEAUti yet:
it allows for using phylogenies of a fixed crown age. 

\end{methods}

%%%%%%%%%%%%%%%%%%%%%%%%%%%%%%%%%%%%%%%%%%%%%%%%%%%%%%%%%%%%%%%%%%%%%%%%%%%%%%%%%%%%%%
\section{Discussion}
%%%%%%%%%%%%%%%%%%%%%%%%%%%%%%%%%%%%%%%%%%%%%%%%%%%%%%%%%%%%%%%%%%%%%%%%%%%%%%%%%%%%%%

beastscriptr does not support the full functionality of BEAUti. Considering
the size, age and number of plugins, this would be close to impossible.
To compensate for this, an extensible software architecture is used.

beastscriptr has minimal support for calling BEAST2 from within R and does
so for testing purposes. 

beastscript has minimal support for parsing and interpreting BEAST2 output files,
for that the RBeast package is recommended.

%%%%%%%%%%%%%%%%%%%%%%%%%%%%%%%%%%%%%%%%%%%%%%%%%%%%%%%%%%%%%%%%%%%%%%%%%%%%%%%%%%%%%%
\section{Conclusion}
%%%%%%%%%%%%%%%%%%%%%%%%%%%%%%%%%%%%%%%%%%%%%%%%%%%%%%%%%%%%%%%%%%%%%%%%%%%%%%%%%%%%%%

beastscriptr successfully creates valid BEAST2 input files. Unlike BEAUti,
beascriptr allows for setting a fixed crown age.

%%%%%%%%%%%%%%%%%%%%%%%%%%%%%%%%%%%%%%%%%%%%%%%%%%%%%%%%%%%%%%%%%%%%%%%%%%%%%%%%%%%%%%
\section*{Acknowledgements}
%%%%%%%%%%%%%%%%%%%%%%%%%%%%%%%%%%%%%%%%%%%%%%%%%%%%%%%%%%%%%%%%%%%%%%%%%%%%%%%%%%%%%%

We would like to thank the Center for Information Technology of the University of Groningen for their support
and for providing access to the Peregrine high performance computing cluster.

%%%%%%%%%%%%%%%%%%%%%%%%%%%%%%%%%%%%%%%%%%%%%%%%%%%%%%%%%%%%%%%%%%%%%%%%%%%%%%%%%%%%%%
\section*{Funding}
%%%%%%%%%%%%%%%%%%%%%%%%%%%%%%%%%%%%%%%%%%%%%%%%%%%%%%%%%%%%%%%%%%%%%%%%%%%%%%%%%%%%%%

This work has been supported by the VICI of Rampal S. Etienne.

%%%%%%%%%%%%%%%%%%%%%%%%%%%%%%%%%%%%%%%%%%%%%%%%%%%%%%%%%%%%%%%%%%%%%%%%%%%%%%%%%%%%%%
\bibliographystyle{natbib}
%\bibliographystyle{achemnat}
%\bibliographystyle{plainnat}
%\bibliographystyle{abbrv}
%\bibliographystyle{bioinformatics}
%
\bibliographystyle{plain}
%
\bibliography{document}


\begin{thebibliography}{}

\end{thebibliography}
%%%%%%%%%%%%%%%%%%%%%%%%%%%%%%%%%%%%%%%%%%%%%%%%%%%%%%%%%%%%%%%%%%%%%%%%%%%%%%%%%%%%%%

\end{document}
